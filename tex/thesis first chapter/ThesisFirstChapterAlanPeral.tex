\documentclass[12pt]{pom_thesis}
\author{Alan Antonio Peral Ortiz}
\advisor{Shahriar Shahriari}
\title{The Parks Location Problem}
\usepackage{tikz}
\usepackage{amsmath}
 
 \theoremstyle{definition}
\newtheorem{definition}{Definition}[section]

\newcounter{mycount}
\newcounter{mycounter}

\tikzset{
	mygrid/.pic={
		  
\draw[step=1cm,color=gray] (-2,-2) grid (3,3);
    \node[fill=cyan] at (-1.5,2.5) {};
    \node[fill=cyan] at (0.5,2.5) {};
    \node[fill=cyan] at (-.5,.5) {};
    \node[fill=cyan] at (2.5,.5) {};
    \node[fill=cyan] at (-1.5,2.5) {};
    \node[fill=cyan] at (1.5,-.5) {};
    \node[fill=cyan] at (-1.5,-1.5) {};

\setcounter{mycount}{1}
\foreach \y in {+2.6,+1.6,.6,-.4,-1.4}
  \foreach \x in {-1.6,-0.6,.4,1.4,2.4}
    \node at (\x,\y)[anchor=north west] {\tiny\arabic{mycount}\addtocounter{mycount}{1}};


	},
	uncoloredgrid/.pic={
	\draw[step=1cm,color=gray] (-2,-2) grid (3,3);

\setcounter{mycounter}{1}
\foreach \y in {+2.6,+1.6,.6,-.4,-1.4}
  \foreach \x in {-1.6,-0.6,.4,1.4,2.4}
    \node at (\x,\y)[anchor=north west] {\tiny\arabic{mycounter}\addtocounter{mycounter}{1}};
	
	
	},
	twoparkschosengrid/.pic={
	\draw[step=1cm,color=gray] (-2,-2) grid (3,3);
   	\node[fill=cyan] at (-1.5,2.5) {};
    	\node[fill=green] at (-.5,.5) {};
     	\node[fill=cyan] at (0.5,2.5) {};
    	\node[fill=cyan] at (-1.5,2.5) {};
    	\node[fill=cyan] at (2.5,.5) {};
    	\node[fill=green] at (1.5,-.5) {};
    	\node[fill=cyan] at (-1.5,-1.5) {};
    	\node[fill=orange] at (-1.5,1.5){};
     	\node[fill=orange] at (-1.5,.5){};
     	\node[fill=orange] at (-1.5,-.5){};
     	\node[fill=orange] at (-.5,1.5){};
     	\node[fill=orange] at (-.5,-.5){};
     	\node[fill=orange] at (0.5,-.5){};
      	\node[fill=orange] at (.5,.5){};
       	\node[fill=orange] at (.5,1.5){};
       	\node[fill=orange] at (1.5,.5){};
       	\node[fill=orange] at (.5,-1.5){};
        	\node[fill=orange] at (1.5,-1.5){};
         \node[fill=orange] at (2.5,-1.5){};
         \node[fill=orange] at (2.5,-.5){};
   
	\setcounter{mycount}{1}
	\foreach \y in {+2.6,+1.6,.6,-.4,-1.4}
  	\foreach \x in {-1.6,-0.6,.4,1.4,2.4}
    	\node at (\x,\y)[anchor=north west] {\tiny\arabic{mycount}\addtocounter{mycount}{1}};
	
	},
	popgrid/.pic = {
	\draw[step=1cm,color=gray] (-2,-2) grid (3,3);
   	\node[fill=cyan] at (-1.5,2.5) {};
    	\node[fill=cyan] at (-.5,.5) {};
     	\node[fill=cyan] at (0.5,2.5) {};
    	\node[fill=cyan] at (-1.5,2.5) {};
    	\node[fill=cyan] at (2.5,.5) {};
    	\node[fill=cyan] at (1.5,-.5) {};
    	\node[fill=cyan] at (-1.5,-1.5) {};
    	\node at (-1.5,1.5){50};
     	\node at (-1.5,.5){45};
     	\node at (-1.5,-.5){40};
     	\node at (-.5,2.5){200};
     	\node at (-.5,1.5){100};
     	\node at (-.5,-.5){50};
     	\node at (0.5,-.5){130};
      	\node at (.5,.5){150};
       	\node at (.5,1.5){150};
       	\node at (1.5, 1.5){160};
       	\node at (2.5, 1.5){20};
       	\node at (2.5,2.5){30};
       	\node at (1.5,2.5){200};
       	\node at (1.5,.5){140};
       	\node at (-.5,-1.5){60};
       	\node at (.5,-1.5){95};
        	\node at (1.5,-1.5){80};
         \node at (2.5,-1.5){55};
         \node at (2.5,-.5){90};
   
	\setcounter{mycount}{1}
	\foreach \y in {+2.6,+1.6,.6,-.4,-1.4}
  	\foreach \x in {-1.6,-0.6,.4,1.4,2.4}
   	\node at (\x,\y)[anchor=north west] {\tiny\arabic{mycount}\addtocounter{mycount}{1}};
	},
	bigexample/.pic = {
		
	\draw [step=0.5cm, very thin, color=black, solid] (0, 0) grid (7.5, 5);
	% row 1 pop
	\node at (0.25,4.75){\footnotesize 50};
	\node at (0.75,4.75){\footnotesize 55};
	\node at (1.75,4.75){\footnotesize 40};
	\node at (2.25,4.75){\footnotesize 60};
	\node at (2.75,4.75){\footnotesize 85};
	\node at (3.25,4.75){\footnotesize 65};
	\node at (3.75,4.75){\footnotesize 120};
	\node at (4.25,4.75){\footnotesize 105};
	\node at (4.75,4.75){\footnotesize 95};
	\node at (5.25,4.75){\footnotesize 85};
	\node at (6.25,4.75){\footnotesize 65};
	\node at (6.75,4.75){\footnotesize 45};
	% row 2 pop
	\node at (0.75,4.25){\footnotesize 50};
	\node at (1.25,4.25){\footnotesize 45};
	\node at (1.75,4.25){\footnotesize 60};
	\node at (2.25,4.25){\footnotesize 110};
	\node at (2.75,4.25){\footnotesize 120};
	\node at (3.25,4.25){\footnotesize 95};
	\node at (3.75,4.25){\footnotesize 110};
	\node at (4.75,4.25){\footnotesize 100};
	\node at (5.25,4.25){\footnotesize 65};
	\node at (5.75,4.25){\footnotesize 75};
	\node at (6.25,4.25){\footnotesize 40};
	\node at (6.75,4.25){\footnotesize 45};
	\node at (7.25,4.25){\footnotesize 30};
	% row 3 pop
	\node at (0.25,3.75){\footnotesize 80};
	\node at (0.75,3.75){\footnotesize 75};
	\node at (1.25,3.75){\footnotesize 55};
	\node at (1.75,3.75){\footnotesize 70};
	\node at (2.25,3.75){\footnotesize 105};
	\node at (2.75,3.75){\footnotesize 130};
	\node at (3.25,3.75){\footnotesize 115};
	\node at (4.25,3.75){\footnotesize 105};
	\node at (4.75,3.75){\footnotesize 115};
	\node at (5.25,3.75){\footnotesize 80};
	\node at (5.75,3.75){\footnotesize 50};
	\node at (6.25,3.75){\footnotesize 65};
	\node at (6.75,3.75){\footnotesize 50};
	% row 4 pop
	\node at (0.25,3.25){\footnotesize 165};
	\node at (1.25,3.25){\footnotesize 65};
	\node at (2.25,3.25){\footnotesize 65};
	\node at (2.75,3.25){\footnotesize 80};
	\node at (3.25,3.25){\footnotesize 95};
	\node at (3.75,3.25){\footnotesize 110};
	\node at (4.25,3.25){\footnotesize 105};
	\node at (5.25,3.25){\footnotesize 70};
	\node at (5.75,3.25){\footnotesize 55};
	\node at (6.25,3.25){\footnotesize 45};
	\node at (7.25,3.25){\footnotesize 25};
	% row 5 pop
	\node at (0.25,2.75){\footnotesize 225};
	\node at (0.75,2.75){\footnotesize 200};
	\node at (1.25,2.75){\footnotesize 70};
	\node at (1.75,2.75){\footnotesize 75};
	\node at (2.25,2.75){\footnotesize 100};
	\node at (3.25,2.75){\footnotesize 100};
	\node at (3.75,2.75){\footnotesize 105};
	\node at (4.25,2.75){\footnotesize 120};
	\node at (4.75,2.75){\footnotesize 105};
	\node at (5.25,2.75){\footnotesize 55};
	\node at (6.25,2.75){\footnotesize 20};
	\node at (6.75,2.75){\footnotesize 25};
	\node at (7.25,2.75){\footnotesize 15};
	% row 6 pop
	\node at (0.25,2.25){\footnotesize 250};
	\node at (0.75,2.25){\footnotesize 120};
	\node at (1.75,2.25){\footnotesize 85};
	\node at (2.25,2.25){\footnotesize 105};
	\node at (2.75,2.25){\footnotesize 100};
	\node at (3.25,2.25){\footnotesize 110};
	\node at (3.75,2.25){\footnotesize 70};
	\node at (4.25,2.25){\footnotesize 80};
	\node at (4.75,2.25){\footnotesize 70};
	\node at (5.25,2.25){\footnotesize 65};
	\node at (5.75,2.25){\footnotesize 50};
	\node at (6.25,2.25){\footnotesize 15};
	\node at (6.75,2.25){\footnotesize 5};
	\node at (7.25,2.25){\footnotesize 10};
	% row 7 pop
	\node at (0.25,1.75){\footnotesize 200};
	\node at (0.75,1.75){\footnotesize 100};
	\node at (1.25,1.75){\footnotesize 180};
	\node at (1.75,1.75){\footnotesize 155};
	\node at (2.25,1.75){\footnotesize 125};
	\node at (2.75,1.75){\footnotesize 105};
	\node at (3.25,1.75){\footnotesize 80};
	\node at (4.25,1.75){\footnotesize 95};
	\node at (4.75,1.75){\footnotesize 75};
	\node at (5.25,1.75){\footnotesize 40};
	\node at (5.75,1.75){\footnotesize 45};
	\node at (6.25,1.75){\footnotesize 35};
	\node at (7.25,1.75){\footnotesize 15};
	% row 8 pop
	\node at (0.25,1.25){\footnotesize 250};
	\node at (1.75,1.25){\footnotesize 145};
	\node at (2.75,1.25){\footnotesize 235};
	\node at (3.25,1.25){\footnotesize 100};
	\node at (3.75,1.25){\footnotesize 80};
	\node at (4.25,1.25){\footnotesize 135};
	\node at (4.75,1.25){\footnotesize 95};
	\node at (5.25,1.25){\footnotesize 70};
	\node at (6.25,1.25){\footnotesize 50};
	\node at (6.75,1.25){\footnotesize 45};
	\node at (7.25,1.25){\footnotesize 20};
	% row 9 pop
	\node at (0.25,0.75){\footnotesize 300};
	\node at (1.25,0.75){\footnotesize 265};
	\node at (1.75,0.75){\footnotesize 205};
	\node at (2.25,0.75){\footnotesize 235};
	\node at (2.75,0.75){\footnotesize 505};
	\node at (3.25,0.75){\footnotesize 500};
	\node at (3.75,0.75){\footnotesize 245};
	\node at (4.25,0.75){\footnotesize 175};
	\node at (4.75,0.75){\footnotesize 80};
	\node at (5.25,0.75){\footnotesize 85};
	\node at (5.75,0.75){\footnotesize 65};
	\node at (6.25,0.75){\footnotesize 30};
	\node at (7.25,0.75){\footnotesize 25};
	% row 10 pop
	\node at (0.25,0.25){\footnotesize 305};
	\node at (0.75,0.25){\footnotesize 255};
	\node at (1.25,0.25){\footnotesize 275};
	\node at (1.75,0.25){\footnotesize 215};
	\node at (2.25,0.25){\footnotesize 225};
	\node at (2.75,0.25){\footnotesize 400};
	\node at (3.25,0.25){\footnotesize 405};
	\node at (3.75,0.25){\footnotesize 410};
	\node at (4.25,0.25){\footnotesize 500};
	\node at (4.75,0.25){\footnotesize 305};
	\node at (5.75,0.25){\footnotesize 45};
	\node at (6.25,0.25){\footnotesize 50};
	\node at (6.75,0.25){\footnotesize 40};
	\node at (7.25,0.25){\footnotesize 30};

	% candidate parcels
	\filldraw[fill=cyan, draw=black] (7, 4.5) rectangle (7.5, 5);
	\filldraw[fill=cyan, draw=black] (5.5, 4.5) rectangle (6, 5);
	\filldraw[fill=cyan, draw=black] (1, 4.5) rectangle (1.5, 5);
	\filldraw[fill=cyan, draw=black] (0, 4) rectangle (0.5, 4.5);
	\filldraw[fill=cyan, draw=black] (4, 4) rectangle (4.5, 4.5);
	\filldraw[fill=cyan, draw=black] (3.5, 3.5) rectangle (4, 4);
	\filldraw[fill=cyan, draw=black] (7, 3.5) rectangle (7.5, 4);
	\filldraw[fill=cyan, draw=black] (0.5, 3) rectangle (1, 3.5);
	\filldraw[fill=cyan, draw=black] (1.5, 3) rectangle (2, 3.5);
	\filldraw[fill=cyan, draw=black] (4.5, 3) rectangle (5, 3.5);
	\filldraw[fill=cyan, draw=black] (6.5, 3) rectangle (7, 3.5);
	\filldraw[fill=cyan, draw=black] (2.5, 2.5) rectangle (3, 3);
	\filldraw[fill=cyan, draw=black] (5.5, 2.5) rectangle (6, 3);
	\filldraw[fill=cyan, draw=black] (1, 2) rectangle (1.5, 2.5);
	\filldraw[fill=cyan, draw=black] (3.5, 1.5) rectangle (4, 2);
	\filldraw[fill=cyan, draw=black] (6.5, 1.5) rectangle (7, 2);
	\filldraw[fill=cyan, draw=black] (0.5, 1) rectangle (1, 1.5);
	\filldraw[fill=cyan, draw=black] (1, 1) rectangle (1.5, 1.5);
	\filldraw[fill=cyan, draw=black] (2, 1) rectangle (2.5, 1.5);
	\filldraw[fill=cyan, draw=black] (5.5, 1) rectangle (6, 1.5);
	\filldraw[fill=cyan, draw=black] (0.5, 0.5) rectangle (1, 1);
	\filldraw[fill=cyan, draw=black] (6.5, 0.5) rectangle (7, 1);
	\filldraw[fill=cyan, draw=black] (5, 0) rectangle (5.5, 0.5);
	
	}
}



\begin{document}

\maketitle

\begin{abstract}We take a look at the multiobjective facility location problem of park placement in the city of Bogot\'{a}, Colombia. We begin with an introduction to operations research, delving into its development and history, before we move on a to simplified example model. We work up from this example to look at the final model. We then examine more information about multiobjective and facility location problems.
\end{abstract}

\pagenumbering{roman}
\tableofcontents

\newpage
\pagenumbering{arabic}
% this is how you begin a chapter:
% the label is so that you can refer to it later
\begin{chapter}{Introduction}
\label{Intro}
\section{What is Operations Research?}

	We begin with a brief history of Operations Research (OR), which as a modern discipline has its roots in the Soviet political economy as well as the British war effort of World War II. 
	
	[SECTION ABOUT SOVIET UNION + OR]
	
	Now, in the year 1934 as Nazi Germany denounced the Treaty of Versailles, Britain sensed the threat and raced to strengthen its defenses. By the following year, radar had been developed and was capable of effectively detecting enemy aircraft, although its utility was stringent on its ability for integration with the existing defense systems: ground observers, interceptor aircraft, and antiaircraft artillery positions. The first task of the newly formed operational research group was to use scientific, rigorous processes to develop a system for the incorporation of radar into existing infrastructure, which was far from the more well-defined mathematical processes that exist today. However, this integrated radar-based air defense system increased the probability of intercepting an enemy aircraft by a factor of ten [CITE HISTORY OF OR IN US MILITARY BOOK]. 
	
	 Soon, the usage of OR spread to investigate other problems, and in 1940 was even called upon to influence high-level strategic policy, when the French requested additional RAF fighter support. Churchill was inclined to acquiesce to the request, but an OR team showed that sending more RAF fighters would weaken Britain beyond recovery in the face of a German attempt to invade Britain [CITE HISTORY OF OR IN US MILITARY]. Churchill was convinced by their presentation, and so he did not send the additional aircraft, preserving the pilots and aircrafts for the Battle of Britain instead. This decision, along with the incorporation of radar, contributed significantly to Britain's victory in the Battle of Britain [CITE HISTORY OF OR IN US MILITARY]. 
	Another significant accomplishment of OR analysts involved the work of deciding upon depth charge settings for bombs. Their work in this field led to an immediate improvement on aerial attacks of German submarines, with estimates of the increased efficiency ranging from 400 to 700 percent [CITE HISTORY OF OR IN US MILITARY]. Other work included reducing the number of artillery rounds required to down one German aircraft  from twenty thousand in the summer of 1941 to merely four thousand the following year [CITE HISTORY OF OR IN US MILITARY].
	From the British, this discipline spread to the Americans, where it was also used in the war effort. [TALK MORE ABOUT 'MURICA HERE!].
	
	There was a specific need to allocate scarce resources to military operations in an effective manner, and so the British and US militaries had scientists perform \textit{research on} (military) \textit{operations}. In essence, the goal was to make the war machine more efficient, and they succeeded. They developed effective ways to use the new radar technology, as well as came up with better ways to manage convoys and conduct antisubmarine operations [CITE INTRO TO OR BOOK]. 
	
	The success that OR saw in the war then encouraged interest in non-military applications of the field.
% more intro shit
	A cursory glance at a variety of introductory textbooks will reveal that there is a certain focus on private sector applications of the field [Should probably cite this claim]. These applications tend to be primarily concerned with profit maximization and other aspects of running a business. While these problems provide for some interesting mathematical formulations, the field's ability to 
	
	There is a subfield of the discipline called Community-Based Operations Research, which seeks to shift the focus of OR from profit maximization or cost-reduction to improving the quality of life within a community. One of the main advocates of the field and of this lens that focuses on people as opposed to money is Michael P. Johnson, who compiled a textbook containing a variety of case studies that can constitute ``Community-Based Operations Research." 
	
	One of these case studies presented is the problem of park location in Bogot\'{a}, Colombia.

In the introduction I will explain the history of operations research and how it tends to be most utilized in the private sector. I will detail the development of (and necessity for) community-based operations research, and then explain that I will be examining a particular case study: the problem of developing new public parks in Bogot\'a, Colombia. I will perhaps provide a literature review here, a roadmap of what I will cover in the rest of the thesis, and anything else that may fit and come to mind later.

\section{An Introduction to the Parks Location Problem}

	In urban areas and large cities, the presence of public parks, green spaces, and other recreational facilities has been associated with a significant improvement in quality of life, mental health, and general wellbeing [CITE SOURCE]. Knowing this, the city of Bogot\'{a} (Colombia), one of the largest cities in Latin America with a population of about eight million that is expected to reach ten million by 2025, has implemented a number of changes that include the recovery of public spaces and the improvement of public parks [CITE THE SOURCE]. 
	
	In 2006, the mayor and city council of Bogot\'{a} threw their support behind a sports and recreation master plan for the city. This plan indicated that by 2019 the city must reach a minimum level of $2.71$  $\textrm{m}^2$ of neighborhood park area per resident. It then became the \textit{Instituto Distrital de Recreaci\'{o}n y Deporte} (IDRD), or Recreation and Sports Institute of Bogot\'{a}'s job to implement the master plan. As such, the IDRD was faced with a monumental challenge: they had to execute the construction of numerous new parks and revitalize dilapidated public spaces in a manner that balanced the differing geographic, social, and economic needs of the city. 
	
	Because of the many needs they had to consider and due to the nature of the problem, this problem was then modeled in such a way that it became a multiobjective facility location problem, which we will examine in further detail at a later point. For now we begin to consider the model.
% you should close a chapter before beginning a new one!
\end{chapter}

\begin{chapter}{A Look at the Model}
\section{A Simplified Model}
In this chapter I present a simplified version of the model. We will take a look at some simpler examples and gradually increase the complexity of our example until we arrive at the one presented in the paper. 

\subsection{Setting up the Problem}

We begin with a $5\times5$ grid, which will represent our simplified fictional city. Each block represents a plot of land, which might either be empty or occupied. If it is occupied, this means that there are people living on the block. Otherwise, the lot is empty and it is a candidate parcel, to potentially be turned into a park. We number each block from one to $25$, and we have colored every candidate parcel cyan. We want to build one park, and so we can choose from amongst all the cyan parcels. This is shown in Figure \ref{fig:grid1}. We note that every lot is the same size. We also define the service area of the candidate parcels to consist of all the populated blocks adjacent and diagonal to the candidate parcel. For example, candidate parcel one's service area will consist of the set $\{2,6,7\}$. In other words, if we built a park on candidate parcel one, then the visitors would be from \textbf{only} those three surrounding lots. The first question we are interested in asking then becomes: 
\begin{center}
\textbf{Which candidate parcel will maximize the geographical coverage, as measured by the service area of the parcel?}
\end{center}
\begin{figure}
 \centering
 \caption{Our $5\times 5$ grid city}
 \label{fig:grid1}
 \begin{tikzpicture}[every node/.style={minimum size=1cm-\pgflinewidth}]
  \pic{mygrid};
\end{tikzpicture}
\end{figure}
In this example, it is possible to answer our question after merely observing the grid. Candidate parcel 12 would result in the highest number of lots served with 8. We now move on to defining a certain number of sets and variables so that we can present a mathematical formulation of the problem. \newline 
	
	We first let $\mathcal{J}$ be the set of all blocks that would benefit from the construction of a new park. Looking back to Figure \ref{fig:grid1}, we see that $\mathcal{J} = \{ 2,4,6,7, \dots, 22, 23, 24, 25\}$. The candidate parcels have not been included in this set, and in this case we note that lot five is not included in $\mathcal{J}$ either because it is outside the service area of every candidate parcel. We also then define the set $\mathcal{I}$ to be the set of candidate parcels. In our case, $\mathcal{I} = \{ 1,3,12,15,19,21\}$. We finally define the set $\mathcal{W}_j$ for $j \in \mathcal{J}$, which consists of all the candidate parcels that service block $j$. In our example, $\mathcal{W}_7 = \{1,3,12\}$. \newline
	
	We now define $z_j$ as a binary decision variable, taking on the value 1 if block $j \in \mathcal{J}$ is covered by at least one park, and 0 otherwise. So if we decided to build a park on candidate parcel 21, then $z_{16} = z_{17} = z_{22} = 1$, and for every other possible value of $j$ we would have $z_j = 0$. We also define the binary decision variable $y_i$, which will take on value 1 if candidate parcel $i$ is selected to become a park, and 0 otherwise. If we turned candidate parcel 21 into a park, then $y_{21} = 1$. We can now formulate our initial model:
\begin{align*}
\textrm{max } f_1 &= \sum_{j \in \mathcal{J}} z_j \\
\textrm{subject to } z_j &\leq \sum_{i \in \mathcal{W}_j} y_i, j \in \mathcal{J}\\
\left|\mathcal{W}_j\right|z_j &\geq \sum_{i \in \mathcal{W}_j} y_i, j \in \mathcal{J} \\
p_{max} &\geq \sum_{i \in \mathcal{I}} y_i \\
z_j &\in \{0,1\}, j \in \mathcal{J} \\
y_i &\in \{0,1\}, i \in \mathcal{I}
\end{align*}

	Our objective function seeks to maximize the geographical coverage of the potential parks to be built. The first constraint guarantees that if block $j$ is covered, then at least one parcel servicing it has been selected as a park. So if block 2 is covered in figure \ref{fig:grid1}, then either candidate parcel 1 or candidate parcel 3 should have been selected to become a park. Conversely, the second constraint guarantees that if block $j$ is not covered, then none of the candidate parcels servicing it should be selected as parks. While I have referred to these first two constraints as being one individual constraint each, the observation can be made that they must be satisfied for all values of $j \in \mathcal{J}$, meaning that each constraint must be repeated $|\mathcal{J}|$ times. \\ \\
	\begin{figure}
	\centering
	\caption{Fully expanded model}
	\label{expandedmodel}
	
		\begin{align*}
		\textrm{max } f_1 &= z_2 + z_4 + z_6 + z_7 + z_8 + z_9 + z_{10} + z_{11} + z_{13} + z_{14} \\
		&+ z_{16} + z_{17} + z_{18} + z_{20} + z_{22} + z_{23} + z_{24} + z_{25}
		\end{align*}
	\begin{equation*}
	\begin{aligned}[c]
	\textrm{subject to } z_2 &\leq y_1 + y_3 \\
	z_4 &\leq y_3\\
	z_6 &\leq y_1 + y_{12}\\
	z_7 &\leq y_1 + y_3 +y_{12}\\
	z_8 &\leq y_3 + y_{12} \\
	z_9 &\leq  y_3 + y_{15} \\
	z_{10} &\leq y_{15} \\
	z_{11} &\leq y_{12} \\
	z_{13} &\leq y_{12} + y_{19} \\
	z_{14} &\leq y_{15} + y_{19} \\
	z_{16} &\leq y_{12} + y_{21} \\
	z_{17} &\leq y_{12} + y_{21} \\
	z_{18} &\leq y_{12} + y_{19} \\
	z_{20} &\leq y_{15} + y_{19} \\
	z_{22} &\leq y_{21} \\
	z_{23} &\leq y_{19} \\
	z_{24} &\leq y_{19} \\
	z_{25} &\leq y_{19} 
	\end{aligned}
	\qquad \qquad \qquad
	\begin{aligned}[c]
	|W_2|z_2 &\geq y_1 + y_3 \\
	|W_4|z_4 &\geq y_3 \\
	|W_6|z_6 &\geq y_1 + y_{12} \\
	|W_7|z_7 &\geq y_1 + y_3 + y_{12} \\
	|W_8|z_8 &\geq y_3 + y_{12} \\
	|W_9|z_9 &\geq y_3 + y_{15} \\
	|W_{10}|z_{10}&\geq y_{15} \\
	|W_{11}|z_{11}&\geq y_{12} \\
	|W_{13}|z_{13}&\geq y_{12} + y_{19} \\
	|W_{14}|z_{14}&\geq y_{15} + y_{19} \\
	|W_{16}|z_{16}&\geq y_{12} + y_{21} \\
	|W_{17}|z_{17}&\geq y_{12} + y_{21} \\
	|W_{18}|z_{18}&\geq y_{12} + y_{19} \\
	|W_{20}|z_{20}&\geq y_{15} + y_{19} \\
	|W_{22}|z_{22}&\geq y_{21} \\
	|W_{23}|z_{23}&\geq y_{19} \\
	|W_{24}|z_{24}&\geq y_{19} \\
	|W_{25}|z_{25}&\geq y_{19}
	\end{aligned}
	\end{equation*}
	\begin{align*}
	p_{max} \geq y_1 + y_3 + y_{12} + y_{15} + y_{19} + y_{21}\\
	z_2, z_4, z_6, z_7, z_8, z_9, z_{10}, z_{11}, z_{12}, z_{13}, z_{14}, \\
	z_{16}, z_{17}, z_{18}, z_{20}, z_{22}, z_{23}, z_{24}, z_{25} \in \{0,1\} \\
	y_1,y_3,y_{12},y_{15},y_{19},y_{21} \in \{0,1\}
	\end{align*}
	\end{figure}
	  The next constraint indicates how many parks we would like to build. The variable $p_{max}$, or the maximum allowable amount of parks we want built, is a constant that is decided upon beforehand. In our case, we decided that we wanted to build one park, so $p_{max} = 1$. We can think of this constraint as a simpler way to encode a budget into our model. Making the simplifying assumption that building a park on any lot will cost the same amount allows us to include this constraint in consideration of our budget. Should our budget change, we can just as easily change the value of $p_{max}$. The last two constraints define our $z_j$ and $y_i$ as binary decision variables.
	  \newline\newline We can see an expansion of the model above to include all true constraints in Figure \ref{expandedmodel}. \\ \\
	As we realized earlier, wanting to select only one candidate parcel will result in the selection of candidate parcel 12. What if we wanted to build two parks? Looking at Figure \ref{fig-grid2} we see that picking the two green parcels would result in a total coverage area of 13 blocks, highlighted in orange, which is the maximum in this case with two candidate parcels. 
	\begin{figure}
	\centering
	\begin{tikzpicture}[every node/.style={minimum size=1cm-\pgflinewidth}]
		\pic{twoparkschosengrid};	
	\end{tikzpicture}
	\caption{Picking two candidate parcels to maximize geographical coverage.}
	\label{fig-grid2}
	\end{figure}
	
	
	We were able to see this result without having to do any calculations because of the simplicity of our model. But this is not always so apparent. We now begin to alter our model and consider other objectives.
	
\subsection{Changing the Objective}

We were previously only concerned with maximizing the geographical coverage of our model. In real life, prioritizing the geographical coverage of the potential parks serves to spread out the location of the parks and avoids building excessive parks in densely populated areas. It also proactively locates parks in places that may have yet to be developed or may be susceptible to rapid population growth in the future. However, we do not want to fully neglect the population density of the various city blocks. How does our model change when we add the number of inhabitants in each block? The question we are interested in now becomes:
\begin{center}
\textbf{How do we maximize the geographical coverage of the parks as well as the number of beneficiaries that would result from the construction of the parks?}
\end{center}

We define the number of beneficiaries from a park construction as the sum of the population of all the blocks in the new park's service area. 

\begin{figure}
	\centering
	\begin{tikzpicture}[every node/.style={minimum size=1cm-\pgflinewidth}]
		\pic{popgrid};
	\end{tikzpicture}
	\caption{City blocks with population numbers.}
	\label{fig:gridpop}
	\end{figure}
	
	We now turn to Figure \ref{fig:gridpop} to see an updated version of our model city. The residential blocks have been updated with a number that indicates the number of people living in that city block. This allows us to calculate the number of beneficiaries that would result from the construction of a park on any candidate parcel. Building a park on candidate parcel 21, for example, would result in 150 beneficiaries. We define one additional variable.
	
	Let $p_i$ be the number of beneficiaries resulting from building a park on candidate parcel $i$. In the example of parcel 21, we would have $p_{21} = 150$. We can now update our model:

\begin{align*}
\textrm{max } f_1 &= \sum_{j \in \mathcal{J}} z_j \\
\textrm{max } f_3 &= \sum_{i \in \mathcal{I}} p_iy_i \\
\textrm{subject to } z_j &\leq \sum_{i \in \mathcal{W}_j} y_i, j \in \mathcal{J}\\
\left|\mathcal{W}_j\right|z_j &\geq \sum_{i \in \mathcal{W}_j} y_i, j \in \mathcal{J} \\
p_{max} &\geq \sum_{i \in \mathcal{I}}y_i \\
z_j &\in \{0,1\}, j \in \mathcal{J} \\
y_i &\in \{0,1\}, i \in \mathcal{I}
\end{align*}

	Our model now has two objective functions. The first objective $f_1$ still seeks to maximize the geographical coverage of the parks. The second objective $f_3$ seeks to maximize the number of beneficiaries. But how do we maximize two things at the same time? We cannot guarantee that there will be a solution that maximizes the number of beneficiaries and the geographical coverage simultaneously. If we consider individual parcels, we notice that candidate parcel 3 serves the most amount of people, with $f_3 = 810$. However, candidate parcel 12 still has the most expansive geographical coverage, with $f_1 = 8$. How do we reconcile these two solutions?
%
%
% section 2.2 - A solution strategy
%
%
\section{A Solution Strategy}

	Because we are now trying to optimize multiple objectives, our question turns into a multiobjective optimization problem. We will present one possible way to solve these types of problems, borrowing the method used by the researchers from Bogot\'{a}. They implemented a lexicographic ordering of the objectives, which in practice meant that they ranked their objectives in accordance with their priorities. These priorities are reflected by the subscripts of the objective functions. Geographical coverage was deemed the most important criterion, which is why it is labeled $f_1$. Maximizing the number of beneficiaries was likewise deemed the third-most important criterion, so it is accordingly labeled $f_3$. For our model these are the only ones that we are concerned with, but it could work with any number of objective functions (for example, six). \\
	The solution strategy is composed of two main parts. The first part is concerned with setting a benchmark for each objective function. The second part is about compromising to find a best possible solution. We will go over both parts in great detail.
%
% 2.2.1 solving in isolation
%	
\subsection{Solving in Isolation}

	For the first part of our solution strategy, we will try to set a benchmark for each objective function by optimizing each of them in isolation. What does this mean? We return to our multiobjective optimization problem to explain the process.
	\begin{align*}
	\textrm{max } f_1 &= \sum_{j \in \mathcal{J}} z_j \\
	\textrm{max } f_3 &= \sum_{i \in \mathcal{I}} p_iy_i \\
	\textrm{subject to } z_j &\leq \sum_{i \in \mathcal{W}_j} y_i, j \in \mathcal{J}\\
	\left|\mathcal{W}_j\right|z_j &\geq \sum_{i \in \mathcal{W}_j} y_i, j \in \mathcal{J} \\
	p_{max} &\geq \sum_{i \in \mathcal{I}}y_i \\
	z_j &\in \{0,1\}, j \in \mathcal{J} \\
	y_i &\in \{0,1\}, i \in \mathcal{I}
	\end{align*}
We see that our model has two objective functions, and we wish to solve for them in isolation. Our aim is to be able to treat each objective function as if it's the only objective function in our linear program. Since we have only two objective functions in this model, then one way to do this is by turning this model into two separate linear programs, as such:
\begin{equation*}
\begin{aligned}[c]
\textrm{max } f_1 &= \sum_{j \in \mathcal{J}} z_j \\
\textrm{subject to } z_j &\leq \sum_{i \in \mathcal{W}_j} y_i, j \in \mathcal{J}\\
\left|\mathcal{W}_j\right|z_j &\geq \sum_{i \in \mathcal{W}_j} y_i, j \in \mathcal{J} \\
p_{max} &\geq \sum_{i \in \mathcal{I}} y_i  \\
z_j &\in \{0,1\}, j \in \mathcal{J} \\
y_i &\in \{0,1\}, i \in \mathcal{I}
\end{aligned}
\qquad \qquad
\begin{aligned}[c]
\textrm{max } f_3 &= \sum_{i \in \mathcal{I}} p_iy_i \\
\textrm{subject to } z_j &\leq \sum_{i \in \mathcal{W}_j} y_i, j \in \mathcal{J}\\
\left|\mathcal{W}_j\right|z_j &\geq \sum_{i \in \mathcal{W}_j} y_i, j \in \mathcal{J} \\
p_{max} &\geq \sum_{i \in \mathcal{I}} y_i  \\
z_j &\in \{0,1\}, j \in \mathcal{J} \\
y_i &\in \{0,1\}, i \in \mathcal{I}
\end{aligned}
\end{equation*}
Splitting our original multiobjective problem into multiple linear programs in this manner allows us to reduce our initial complicated problem into multiple, familiar, subproblems. We then solve each individual linear program, and we get the best possible result for each objective. Referring back to Figure \ref{fig:gridpop}, we again see that the first linear program will give us the result $f_1 = 8$, because that is the maximum geographical coverage possible. The second linear program will give us the result $f_3 = 810$, because as we established earlier, this is the maximum amount of beneficiaries served by any one candidate parcel. We will henceforth refer to these optimal values solved in isolation as $f_1^*$ and $f_3^*$, with $f_1^* = 8$ and $f_3^* = 810$ as stated. \\

Performing this process, however, brings us no closer to finding a suitable solution for both objective functions simultaneously, because as we saw, each	optimal solution for the linear programs results in different candidate parcels being picked (parcels twelve and three, respectively). What we have now, though, are figures that serve as an upper bound and a benchmark for our multiobjective problem. We know that, given no other considerations, when we want to build one park we can serve at most eight lots. Similarly, we know that when only considering the amount of people living in surrounding lots, we can serve at most 810 of them by building one park. In a problem that seeks to simultaneously maximize both of these figures, we can do no better than the individual optimal solutions we found. \\

In fact, this is a process that can be done for any number of objective functions. In our case, we are only working with two objectives, and so we split our program into two linear programs. With $n$ objective functions, we would then accordingly split our multiobjective problem into $n$ linear programs. We would then get an optimal individual solution for each objective, which would function as our best case scenario. In the final solution to our multiobjective problem, because we are trying to maximize both the geographical coverage and the number of beneficiaries served, then we want our solution to approach both optimal values, because as we noticed, we cannot satisfy both optimal values with the ``budget" to build only one park. \\

We can now move onto the second stage of this process, where we present a methodical way to reconcile our solutions and arrive at an ideal middle ground.
%
% 2.2.2
%
\subsection{Guess and Check}

Before we dive into the second stage of the method, we must first define a term that will help us understand our process. 

\begin{definition}{Deterioration Rate}
For some value of the objective function $f_k$, the deterioration rate indicates how much worse our solution is than the optimal solution $f_k^*$ as a percentage. In particular, we are interested in $\alpha_k$, which will be the maximum allowable deterioration for objective $k$.
\end{definition}

Defining this idea of a deterioration rate now allows us to compare all of our solutions against the optimal solution derived from solving in isolation. For example, returning to our parks model, maximizing the geographic coverage alone resulted in a coverage of eight blocks. But maybe optimizing the population and the geographic coverage simultaneously might only result in a geographic coverage of six blocks. This six block solution would then be said to have a deterioration rate of $25\%$. If our solution for optimizing both the population and geographic coverage resulted in a solution with a geographic coverage of eight blocks, then our deterioration rate for the geographic coverage would be $0\%$. Finally, to see how every objective can have a deterioration rate, if our solution for the two objectives happened to be 5 blocks and 680 people respectively, then these solutions would indicate deterioration rates of $37.5\%$ and $16\%$. \\

We can now define a value that is closely related to the deterioration rate.

\begin{definition}{Compromise Threshold}
The compromise threshold indicates how ``optimal'' we want our solutions to be as a percentage of $f_k^*$. In particular, this is expressed as the value $(1-\alpha_k)$.
\end{definition}

What this idea tells us is that for any solution that optimizes multiple objectives, we want it to give a value for $f_k$ that is at least as good as some percent of $f_k^*$. If $f_k$ was an objective that we really cared about, we might be more strict in demanding that any solution be at least $95\%$ of the optimal value $f_k^*$. If we were less strict, we might be more lenient with our solution and say that it would only have to exceed $50\%$ of the optimal solution $f_k^*$. This means that we can define different percentages for different objective functions according to how important we deem them to the problem at hand. Returning to our parks problem, if we wanted any compromise solution to satisfy at least $80\%$ of the optimal value $f_1^*$, then our compromise threshold would be $1 - \alpha_k = 0.8$ and the maximum allowable deterioration would be $\alpha_k =  0.2$. \\

Why are these terms important? Well in fact, they are essential to stage two of our solution strategy, which I will explain first in general terms and then use our parks problem as a guiding example. \\

Remember that earlier, we had decided on ranking our objectives by order of importance. This is what the subscript in $f_k$ denoted. Now, after we have solved for our objectives in isolation, we will need to solve some more linear programs to bring our many different optimal solutions together. We will start with the second-most important objective function, and modify the linear program that we solved in isolation for it. We will actually add a new constraint, which will look like the following: $f_1 \geq (1-\alpha_1)f_1^*$. What does this mean? Remember that $f_1$ was our original most-important objective function. On the right hand side, we have our compromise threshold multiplied by the optimal value of the objective solved in isolation. So what this constraint does is ensure that the value of our objective does not drop below a certain target percentage of the optimal value. And in fact, this second stage of the process involves repeating this for every objective function. For the third-most important objective function and its linear program, we would add both $f_1 \geq (1-\alpha_1)f_1^*$ and $f_2 \geq (1-\alpha_2)f_2^*$ as objective functions. We would continue like this, moving on to the next linear program and turning the previous objective function into a constraint, until we get to our least important objective function, and every previous objective has been added as a constraint, along with the original constraints.\\

Returning to our parks problem to see an example of this strategy, we first note that we only have two objective functions, so our process will be much shorter. In fact, we have only one constraint, $f_1 \geq (1-\alpha_1)f_1^*$, that we will add to the linear program with the third objective function. This constraint will actually look like this: $\sum_{j \in \mathcal{J}} z_j \geq (1-\alpha_1)8$, because $f_1 = \sum_{j \in \mathcal{J}} z_j$ and $f_1^* = 8$. Remember that $\alpha_1$ is our predetermined maximum acceptable deterioration rate. So our new linear program will look like this:

\begin{align*}
\textrm{max } f_3 &= \sum_{i \in \mathcal{I}} p_iy_i \\
\textrm{subject to } z_j &\leq \sum_{i \in \mathcal{W}_j} y_i, j \in \mathcal{J}\\
\left|\mathcal{W}_j\right|z_j &\geq \sum_{i \in \mathcal{W}_j} y_i, j \in \mathcal{J} \\
p_{max} &\geq \sum_{i \in \mathcal{I}} y_i  \\
\sum_{j \in \mathcal{J}} z_j &\geq (1-\alpha_1)8 \\
z_j &\in \{0,1\}, j \in \mathcal{J} \\
y_i &\in \{0,1\}, i \in \mathcal{I}
\end{align*}

To recap, in this linear program we try to optimize for the number of beneficiaries while ensuring that our geographical coverage does not drop below a certain percentage. And in fact, the solutions will change depending on what value for $\alpha_1$ we decide to use. So how can we decide what value of $\alpha_1$ will yield an ideal solution, and in general, how can we decide what values for any $\alpha_k$ to use? This is where automation will come in handy. Since it may not be particularly evident which value of $\alpha_1$ will be the most optimal, we can employ a computer program to try many different values of $\alpha_1$ for us. I devised a program to carry out precisely this task, and in Table \ref{alpha-table-1} we can see the result of running this program. The table has four columns corresponding to the compromise threshold, the maximum allowable deterioration, the number of beneficiaries served (the value solved for $f_3$), and the number of blocks served in the solution (the value for $f_1$). We note that the last two columns have percentage values in parenthesis. These figures express the solution as a percentage of the optimal solutions $f_k^*$ solved in isolation ($810$ for beneficiaries and $8$ for geographic coverage). In our case of the small $5 \times 5$ town, we notice that there seem to be a set of three different, optimal solutions. When dealing with multiple objectives, however, our definition of optimal becomes a little bit muddled. Looking at the chart, we first note that the program tells us that if we relax the constraint on geographic coverage only a little bit by letting $\alpha_1 = 0.1$, the best solution is still to optimize the geographic coverage fully. This then results in $715$ people being served by that selected candidate parcel. However, if we relax our constraint a little bit more and let $\alpha_1 = 0.15$, our solution changes. The number of blocks served drops to $7$, which is not as good as our previous solution, but now the number of people served by the potential park jumps to $740$. Relaxing the geographic constraint a significantly higher amount by letting $\alpha_1 = 0.4$ then results in another decrease in blocks serviced with five, but it has a corresponding jump of people serviced to $810$, which is the maximum amount of people we could service by building one park. With this set of solutions, it seems to be the case that as we decrease our expectations of how many blocks we should service, we are able to service more people. \\

\begin{table}[]
\centering
\hspace*{-1.9cm}\begin{tabular}{llll}
\hline
$(1-\alpha_1)$ & \begin{tabular}[c]{@{}l@{}}Allowable \\ deterioration ($\alpha_1$) \end{tabular} & \# beneficiaries served (\% of max) & \# blocks served (\% of max) \\ \hline
0.9   & 0.1                                                                                    & 715 (88.3)                  & 8 (100)                    \\
0.85  & 0.15                                                                                    & 740 (91.4)                  & 7 (87.5)                    \\
0.8   & 0.2                                                                                    & 740 (91.4)                  & 7 (87.5)                    \\
0.75  & 0.25                                                                                    & 740 (91.4)                  & 7 (87.5)                    \\
0.7   & 0.3                                                                                    & 740 (91.4)                  & 7 (87.5)                    \\
0.65  & 0.35                                                                                    & 740 (91.4)                  & 7 (87.5)                    \\
0.6   & 0.4                                                                                    & 810 (100)                  & 5 (62.5)                    \\
0.55  & 0.45                                                                                    & 810 (100)                   & 5 (62.5)                    \\ \hline
\end{tabular}
\caption{How changing the value of $\alpha_1$ changes the optimal solution}
\label{alpha-table-1}
\end{table}

Since we don't actually have any more objective functions, then stage two of the process is over. We have no other function to repeat it for, and we cannot add $f_3$ as a constraint to a new linear program because we would have no new objective function to optimize. We run into another problem, however, as there seems to be no clear best solution out of the ones given. No one solution is better than the others in every single way, as there is always some tradeoff. This is where domain specific knowledge influences our decision, and it is ideal to consult the urban planners and other interested parties to see which solution is ideal to them. What is more valuable? Having one potential park serve $25$ people extra, or having a potential park service one more block? Do we want to serve as many people as possible at significant cost to the geographic coverage? It's quite hard to say, and often, there won't always be a clear-cut solution to multiobjective problems. What we can do is present a series of ideal solutions, and leave the experts and involved parties to pick the best one in accordance with their experience and criteria for selection.
%
%
%
\subsection{A More Complicated Example}

With our previous example, we were only looking at a small town with $25$ total blocks, and a small number of candidate parcels. It turned out to be a very simple problem to eye-ball, and it was feasible to calculate the geographic coverage and population totals for every single candidate parcel and compare them that way. But we know that towns may be much much bigger. So what if we wanted to look at a bigger area of land? Perhaps a part of a town, or a city district? It would be near impossible to brute force our solutions for much bigger plots of land. This is where the program I designed comes into play, and it can solve much more difficult problems. Consider Figure \ref{bigexample}


\begin{figure}
\centering
\caption{A Daunting Example.}
\label{bigexample}
\begin{tikzpicture}[every node/.style={minimum size=0.5cm-\pgflinewidth}]
  \pic{bigexample};
\end{tikzpicture}
\end{figure}
%
% 2.3 the real model
%
\section{The Real Model}


\end{chapter}

\begin{chapter}{Multiobjective Facility Location Problems}
\section{What?}
Here we will look at the general class of Multiobjective facility location problems, which is the class of problems that contains the main model/problem I will be looking at for my thesis. 
\end{chapter}

\begin{chapter}{Discrete Facility Location Problems}
\section{What? pt. II}
The problem I will be examining also falls under this category. I am not sure how much this would vary from the previous chapter. But if it turns out there is a significant difference between both types of problems, it might be helpful to provide a dedicated chapter for both. Maybe the only difference is that the other class of problems has many objective functions.
\end{chapter}

\begin{chapter}{Applications}
If I have time to apply what I have learned to another problem, maybe this will go here. But this may be a bit ambitious. Stay tuned.
\end{chapter}

\end{document}

 